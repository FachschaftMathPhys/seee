\documentclass[a4paper,10pt]{article}
\usepackage[margin=3cm]{geometry}
\usepackage[english]{babel}
\usepackage{helvet}
\usepackage[utf8]{inputenc}
\pagestyle{empty}
\begin{document}
\subsection*{How to evaluate}
Voluntary or not, you have the privilege to evaluate a lecture.  This is actually quite easy and can even bring some fun. If you pay attention to the following aspects, nothing can go wrong and the people analysing the evaluation will be grateful.

\subsection*{Twelve steps towards a successful evaluation}
\begin{enumerate}
\item Read the instructions \textbf{carefully} and \textbf{in advance}
\item Go to the \textit{Fachschaftsraum} early enough before the lecture starts and collect the folder with the name of your lecture!
\item Go to the lecturer before the lecture and introduce yourself as a member of the \textit{Fachschaft}.
\item The lecturer has already been informed that you will come -- however, remain polite if he/she doesn't know anything about the evaluation.
\item Carry out the evaluation at the beginning of the lecture (or whatever you agreed on with the prof)
\item Explain in a short speech the following aspects. Use the microphone if necessary.
    \begin{itemize}
    \item Why evaluate? Feedback to the professor, improvement of teaching.
    \item Who makes the evaluation? The Fachschaft in cooperation with the \textit{Studienkommission}.
    \item Mark your answer with a \textbf{clear cross} (use a ballpoint pen or pen, no light-coloured fine liners or pencil)
    \item Where can you find the results? In the \textit{Fachschaftsraum} and the institutes.
    \item Mention the automatic analysis of the evaluation! -- Remarks on the sheets are not considered unless put into correct spaces
    \item Fill in the description fields in \textbf{key words} and write clearly. Iteams should speak for themselves, i.e. ``script'' isn't a good comment while ``good script'' or ``script would be nice'' is.
    \end{itemize}
\item Hand out the questionnaire and answer questions.
\item Give enough time to fill in the questionnaire! If you notice that the lecturer starts to become nervous, ask the students whether they still need time or not. During the evaluation there is no lecture!
\item Collect all the sheets! It is very important to not allow the students to hand them in later!
\item Thank the lecturer and say goodbye.
\item Back in the \textit{Fachschaftsraum}, sort out and \textbf{destroy empty sheets}: divide them into halves with the cutting machine.
\item Put the filled-in sheets into the cardboard „filled in sheets“ which you can find by looking for it.
\item Finished!
\end{enumerate}


\subsection*{amusing problems}
\noindent\textbf{not enough sheets:} Go immediately to the Fachschaftsraum or call 06221 544167 and print the sheets or have someone print them! They can be found:\hfill You mustn’t use any other sheets!\\
\textit{§§§/“name of the lecture -- name of the professor”.pdf}\\[0.2cm]
\noindent\textbf{multiple lecturers:} Hand sheets one \emph{after} the other. Write on the board, who is currently evaluated.\\[0.2cm]
\noindent\textbf{one study group, more than one teacher:} Hand out one sheet per teacher. The questions concerning the lecture should only be filled in once.

\end{document}
