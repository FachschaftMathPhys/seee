Im \semesterlong{} wurde in Veranstaltungen der Fakultäten für
\facultylong{} eine Veranstaltungsevaluation durchgeführt.
Dazu wurden die Studenten\footnote{Hiermit sind ausdrücklich auch weibliche
Studenten gemeint, wir werden aber im gesamten Dokument das generische
Maskulinum verwenden.} in den Veranstaltungen gebeten, einen von der
Fachschaft in Zusammenarbeit mit den Studienkommissionen der Fakultäten
entworfenen Fragebogen auszufüllen. 
% FIXME?
Deutsche und Englische Fragebogenversionen
unterscheiden sich einzig in der Sprache, damit auch
Promotionsstudenten ohne deutsche Sprachkenntnis die Möglichkeit
haben, die Veranstaltung zu beurteilen. 
Den Studenten wurde während der Veranstaltung Zeit zur Beurteilung der
Veranstaltung und ggf. des Übungsbetriebs gegeben. Direkt im Anschluss daran wurden die Bögen wieder
von Fachschaftsmitgliedern eingesammelt. Die Fragebögen wurden von
aktiven Mitgliedern der Fachschaft ausgewertet.


\subsubsection{Erläuterungen zur Auswertung}
% FIXME: Seminarbogen?
Durch die erste Frage (\glqq Welcher Anteil der Studenten, die am Anfang an
der Vorlesung teilgenommen haben, ist deiner
Schätzung nach noch dabei?\grqq{}) kann die 
Zuverlässigkeit der Auswertung beurteilt werden. Ist der Anteil der
Studenten zu klein, die noch an 
der Vorlesung teilnehmen, muss oft davon ausgegangen werden, dass
die Ergebnisse verfälscht sind. Man kann annehmen, dass ein großer Anteil 
der Abbrecher den Besuch wegen aus ihrer Sicht schlechter Präsentation
des Vorlesungsstoffes oder ähnlichen Punkten aufgegeben hat. Die Aussage der 
verbliebenen Studenten kann dann nicht mehr als repräsentativ für die 
Zielgruppe der Lehrveranstaltung angesehen werden. 

\subsubsection{Übungsgruppen}
Nach dem allgemeinen Vorlesungsteil der Auswertung folgt eine Auflistung aller Tutoren zu denen Bögen abgegeben wurden. Aufgrund der fehlenden Statistik wurde die Auswertung nur für Tutoren vorgenommen, bei denen mehr als ein Bogen abgegeben wurde. Die Bewertung der Tutoren geht immer von den für die entsprechende Frage abgegebenen Bögen als Maximalwert aus.

\subsubsection{Die Diagramme}

Die Histogramme hinter den Fragen entsprechen den Feldern auf den Fragebögen. 
Die Balkendiagramme geben die Häufigkeit wieder, mit der das entsprechende 
Feld angekreuzt wurde. Die Höhe der Balken ist dabei auf die Gesamtzahl von 
gültigen Stimmen normiert.

Die Mittelwertberechnung zu jeder Frage und Veranstaltung ergibt sich aus den 
jeweils gültigen Stimmen und ist durch den oberen (ausgefüllten)
Punkt dargestellt. Der untere (unausgefüllte) Punkt stellt das Mittel
über alle abgegebenen Bögen der Vergleichsgruppe bei der jeweiligen Frage dar. Dabei wurden alle Bögen gleich gewichtet (kleinere Veranstaltungen
fallen also schwächer ins Gewicht als große). Bei den Fragen zu den
einzelnen Übungsgruppenleitern sind als Vergleich alle Übungsgruppen
der Veranstaltung und nicht der Fakultät als Vergleich heran gezogen. Berücksichtigt
wurden natürlich nur Bögen des aktuellen Semesters und der gleichen
Fakultät -- die zeitgleich ausgeführte Evaluation an der Fakultät
für Physik und Astronomie beeinflusst die Auswertung also
nicht. Der jeweilige Balken gibt die Standardabweichung an. 
Wenn in Vorlesungen Übungsgruppen einzeln aufgeführt werden, ist der
Mittelwert bei den Fragen zum Übungsbetrieb zunächst über alle
Gruppen, bevor anschließend jede Gruppe für sich ausgewertet wird. 

Der Vergleich zwischen dem
Mittelwert einer Veranstaltung und dem  
Gesamtmittel ist in vielen Fällen aussagekräftiger als der Absolutwert des 
Mittelwertes. Es sollte aber beachtet werden, dass sich die Veranstaltungen 
aufgrund zum Teil völlig unterschiedlicher Situationen manchmal nicht 
vergleichen lassen.

\subsubsection{Beispiel}


   \parbox[t]{8.3cm}{\raggedright Wie ist das durchschnittliche Tempo der Vorlesung für dich?}
   \hspace{0.3cm}\rule[-1cm]{0mm}{1cm}
   \smash{\raisebox{-1mm}{\parbox{1.7cm}{\flushright\sffamily\small viel zu hoch}}}
   \smash{\raisebox{-0mm}{\parbox[t]{45mm}{
      \setlength{\unitlength}{0.09mm}   %%Beginn eines Histogramms
      \begin{picture}(500,65)
          \put(0,-75){\framebox(500,100){}}
          \put(238.89,-90){\circle*{15}}
          \put(238.89,-90){\line(1,0){31.43}}
          \put(238.89,-90){\line(-1,0){31.43}}
          \put(207.46,-97){\line(0,1){14}}
          \put(270.32,-97){\line(0,1){14}}
          \put(224.09,-110){\circle{15}}
          \put(224.09,-110){\line(1,0){80.19}}
          \put(224.09,-110){\line(-1,0){80.19}}
          \put(143.9,-117){\line(0,1){14}}
          \put(304.28,-117){\line(0,1){14}}
          \put(0,-75){\rule{9mm}{0mm}}
          \put(100,-75){\rule{9mm}{1mm}}
          \put(100,-75){\line(0,1){100}}
          \put(200,-75){\rule{9mm}{8mm}}
          \put(200,-75){\line(0,1){100}}
          \put(300,-75){\rule{9mm}{0mm}}
          \put(300,-75){\line(0,1){100}}
          \put(400,-75){\rule{9mm}{0mm}}
          \put(400,-75){\line(0,1){100}}
      \end{picture}  %%Ende eines Histogramms
   }}}
   \smash{\raisebox{-1mm}{\parbox{1.7cm}{\flushleft\sffamily\small viel zu niedrig}}}

\subsubsection{Zu den Kommentaren}
In der Vergangenheit wurde wiederholt bemängelt, dass die Kommentare
der Studenten ungekürzt und unzensiert abgedruckt wurden. Deshalb
sind wir dazu übergegangen, die Kommentare zusammenzufassen und ein Bild der Statistik wiederzugeben. Dadurch soll auch verhindert werden, dass die Kommentare einzelner Studenten in krassem
Missverhältnis zur Statistik aller Studenten überbewertet werden.

\subsubsection{Allgemeine Bemerkungen zur Beurteilung der Lehrveranstaltungen}
Die Mittelwerte über alle Veranstaltungen liegen meist höher als
die neutralen Werte. Die Veranstaltungen scheinen von den Studierenden
geringfügig besser als eine durchschnittliche Veranstaltung empfunden zu werden.
Wie oben schon erwähnt muss noch der Anteil der Studierenden
beachtet werden, die den 
Besuch der Veranstaltungen im Laufe dieses Semesters aufgegeben haben und 
deshalb nicht von der Umfrage erfasst worden sind. Von 
diesen ist i.A. eine schlechtere Bewertung zu erwarten. 
Die Evaluation soll dazu dienen Anhaltspunkte für die Verbesserung der Lehre zu geben. Sie ist von den Dozenten und Tutoren als konstruktive Kritik zu sehen. Die Aussagekraft und Notwendigkeit dieses studentischen Meinungsbildes sollte von allen Fakultätsmitgliedern anerkannt werden.

